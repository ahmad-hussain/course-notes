\documentclass[11pt]{report}
\usepackage[margin=0.5in]{geometry}
\usepackage{mathpazo}
\usepackage{amsmath}
\usepackage[colorlinks=true]{hyperref}
\usepackage{parskip}
\usepackage{tikz}
\usepackage{circuitikz}
\tikzstyle{Si}=[circle,draw=blue,fill=blue!50,thick]
\usepackage{listings}
\newcommand{\imp}[1]{\textbf{#1}}
\newcommand{\idate}[2]{\textcolor{blue!50}{\imp{#1}}\label{date:#2}}
\newcommand{\bl}{\begin{itemize}}
\newcommand{\kl}{\end{itemize}}
\newcommand{\song}[2]{\textcolor{red!70}{\textbf{#1}} (\idate{#2}{#1})}

\lstset{numbers=left}

% Make lists condensed %
\usepackage{enumitem}
\setlist{nosep}

\begin{document}

% \chapter{The Big Band Era}
% \chapter{The Blues}
% \chapter{Technological Developments}
% \chapter{The Beginning of Rock 'n' Roll}
% \chapter{The Golden Age of Rock 'n' Roll}
% \chapter{The In-Between Years}
\chapter{The Beatles}

\begin{itemize}
	\item JFK assassinated \idate{Nov 22, 1963}{jfk}
	\item Postwar Britain under the Marshall plan begins to discover American rock 'n' roll
	\bl
		\item Britain's first attempt at rock and roll: \imp{skiffle}
		\bl
			\item Very DIY
			\item Covers of old american blues, e.g. leadbelly
		\kl
	\kl
	\item The Beatles form
	\bl
		\item John Lennon in a skiffle group called the Quarrymen
		\item John meets Paul McCartney after the performance, they bond over the fact that they are both unique in that they actually \textit{write} their own songs, which was not common - they decide to form a band
		\item George Harrison joins as guitarist
		\item Early Beatles had a rocker/motorcycle gang image
		\item They do a lot of live shows, in East Germany etc.
	\kl
	\item The Beatles meet Brian Epstein
	\bl
		\item As manager, he changes their image to be clean and approachable - this is where the main image of the early Beatles comes from
		\item Unable to get a record deal from any label in Britain even after trying for 6 months - this is the in-between years and few people see a future in guitar based music
	\kl
	\item Beatles sign to EMI \idate{1962}{beatles-emi}
	\bl
		\item Producer George Martin
		\bl
			\item He had access to a lot of variety in production equipment - used to produce comedy shows
			\item Tells the Beatles to fire the drummer Pete Best - they do it instantly, probably because they didnt like him that much (especially John)
			\bl
				\item He didn't do drugs
				\item He looked better than John
			\kl
			\item Ringo joins as drummer - Martin is initially unsure and has him play tambourine instead on \song{Love Me Do}{1962}
		\kl
	\kl
	\item The Beatles record and release \song{Please Please Me}{1963}
	\bl
		\item Ringo finally playing on drums
		\item AABA structure
		\item Change in lyrical density for the B section, similar to Somwhere Over the Rainbow
		\item Highest note appears at the end of the B section
	\kl
	\item By the end of \idate{1963}{beatles-popular}, the Beatles are massively popular
	\bl
		\item November 1963, they perform a show at the Royal Variety Performance and John disses the Queen
		\item They get a big hit with 'loves you yeah yeah', and decide to go to America. EMI markets it with the phrase 'the Beatles are coming' (joke on the British are coming)
	\kl
	\item Beatles arrive in America \idate{1964}{beatles-america}
	\bl
		\item Their first stop is the Ed Sullivan show, which gets \imp{70M} viewers (over a third of the US population at the time)
		\item Next stop is the Washington Coliseum
		\bl
			\item At the time, it was unusual to do a concert in a sports stadium; no one was that big
			\item Decide to put the band in the middle of the stadium, turning 90$^\circ$ every once in a while to face different members of the audience
			\item New equipment has to be invented to support these concerts, but they are massively successful. All of a sudden sports stadium owners realise they could be making a whole lot more money in off-season.
		\kl
		\item In 2 weeks, they sell 2 million albums and \$2.5M in merchandise
	\kl
	\item The Beatles become the template for what is to follow, for example the \imp{Mersey Beat} genre of British Blues
	\bl
		\item Gerry and the Pacemakers, The Searchers, and The Swinging Blue Jeans - they all play simple blues with a similar friendly look as the Beatles
	\kl
	\item In \idate{1965}{help-movie}, they release their second film entitled 'Help', essentially a mockumentary of the touring life
	\item \song{Yesterday}{1965}
	\bl
		\item Uses a string quartet - a signifier of serious composers
		\item More complex harmony and lyrics than their previous songs
		\item George's idea for the arrangement - strings and guitar
	\kl
	\item The Beatles meet Dylan
	\bl
		\item The Beatles revered Dylan
		\item He introduces them to weed and not writing terrible lyrics
	\kl
	\item Paul begins to write songs on his own: until now it was always Lennon and McCartney
	\item Beatles begin to move away from themes of idealized romance, and begin to experiment even with album art (e.g. Rubber Soul)
	\bl
		\item Rubber Soul -  cover doesn't even have the band's name on it. The story behind the art is that they accidentally distorted a normal picture and just decided to go with it. The name of the album refers to soul music and how their songs sounded like a rubber imitation of it.
		\item Album art becomes part of the art of the music and no longer just a marketing tool
	\kl
	\item Beatles release the album Revolver (\idate{1966}{beatles-revolver}); on it is the song \song{Tomorrow Never Knows}{1966}
	\bl
		\item Composed by John, lyrics adapted from the Tibetan book of the dead
		\item Influence of non-Western music can be heard in the usage of \textbf{drone}
		\item Double tracked vocals (sung twice on the same track)
		\item \textbf{Leslie cabinet} effect: they played back Johns voice on a spinning cabinet, making it sound like he is singing through a fan
		\item Paul has been experimenting with tape loops, which was an avant-garde technique at the time
		\bl
			\item Literally tape two ends of a recording together and attach it to a machine that plays it
			\item Seagull sound is a sped up version of Paul laughing
			\item There are about 8 loops playing in the song, physically operated by different people
			\item Played the guitar loop backwards
		\kl
	\kl
	\item The Beatles become disillusioned with live performance
	\bl
		\item Philippines tour, they reject dinner with the dictatorial president, becomes a controversy and the Beatles are forced to turn over their concert profits to the government and are harassed by officials
		\item John makes the 'Beatles are bigger than Jesus' comment and it causes a huge stir
		\item Candlestick park, San Francisco \idate{1966}{beatles-last-live} is their last performance, after which they announce they will no longer perform live.
	\kl
\end{itemize}


\chapter{British Invasion}
\section{The End of the Beatles}
\begin{itemize}
	\item Beatles release the single \song{Strawberry Fields Forever}{1967}, their first single since their last performance at Candlestick park
	\item Beatles release the album Sgt. Peppers Lonely Hearts Club Band \idate{1967}{sgt-peppers}
	\bl
		\item Example song \song{A Day In the Life}{1967}
		\bl
			\item Assembled by 2 different partially completed songs written by Lennon and McCartney, an unusual choice
			\item Transition is \textbf{aleatory} which means 'chance' - they let the musicians do whatever, progressively getting louder then playing an E.
		\kl
		\item First \textit{major} concept album (not the first)
		\bl
			\item A concept album is an album with some common theme or story connecting the songs
			\item Originally the album was supposed to be all about childhood, though they ended up adding other songs to pad it out
		\kl
		\item First time that the lyrics were included on the cover of a rock album
		\item Hippie Aesthetic
		\bl
			\item Rock 'n' Roll - entertainment/hit focused music (e.g. Chuck Berry)
			\item Rock - focus on the musician as an artist, singles become albums, dancing becomes listening, etc.
		\kl
	\kl
	\item The end of the Beatles
		\bl
			\item Brian Epstein dies of overdose - beginning of the end for the Beatles
			\item Epstein often considered the '5th Beatle', and was the only guy keeping Paul and John on good terms
			\item Paul leaves the band in \idate{1969}{beatles-end}
		\kl
\end{itemize}

\section{British Blues Revival}
\begin{itemize}
	\item Chess Records does a tourn in the U.K. with Muddy Waters \idate{1958}{chess-records-tour}
	\bl
		\item Why? R\&B was becoming less popular in the in-between years in the USA but the UK was just discovering it
		\item This tour becomes the major influence for a subculture of british blues artists: Fleetwood Mac, Cream, Clapton, Zeppelin, and the Rolling Stones
	\kl
	\item Rolling Stones debut in \idate{1962}{rolling-stones-debut}
	\bl
		\item Founder: \textbf{Brian Jones}, wants to make a band that just does old American blues covers.
		\item Manager: \textbf{Andrew Long Oldham}
		\bl
			\item Tells them to stop aping the Beatles
			\item Gives them their iconic image of ragged dress and mean attitudes
		\kl
		\item \idate{1963}{rolling-stones-bad} - they change into an intimidating image
		\item Oldham pushes them to write their own songs to get more money (contrary to Jones' intention)
		\bl
			\item At one point, brings in Lennon and McCartney to finish writing a song for them, allowing Keith and Mick learn how to write songs
		\kl
		\item \song{King Bee}{1964}
		\bl
			\item Cover of a Slim Harpo song from '57
			\item Does an American accent, which is unusual for british bands at the time
		\kl
		\item They tour in the US, not getting much success. They had not yet begun to write their own songs.
		\item Their first big original is \song{Satisfaction}{1965} which was written in Keith's sleep. The lyrics are about American advertising. It becomes their first \#1 hit in the US.
		\item Brian Jones doesnt like the direction of the band, leaves in 1969 after a dispute. Dies in his pool after they leave for another US tour.
	\kl
	\item Similarities Between and Rolling Stones
	\bl
		\item Managers created their image (RS Oldham, Beatles Epstein)
		\item Opposite trajectories - Beatles for the middle class, RS was 'for the working-class'. But they came from opposite directions - Mick was fairly upper class while the Beatles were not
		\item Product vs. Process
		\bl
			\item Having a goal in mind when going into the studio vs. creating something as a byproduct of experimenting
			\item Beatles were product oriented, Rolling Stones process oriented.
			\item Improvisation = process
		\kl
	\kl
\end{itemize}

\chapter{Soul and Funk}

\begin{itemize}
	\item In the US, especially in Black communities, the tables begin to turn. Influenced by the civil rights movement there was a new vision that rejected R\&B as the music of the past.
	\item Soul
	\bl
		\item Three main characteristics
		\begin{enumerate}
			\item Gospel-Influence Vocals but not lyrical content (acrobatic, strong sense of rhythm)
			\item Rhythym of R\&B
			\item Arrangements and lyrical styles of TPA (to restrain the themes of R\&B)
		\end{enumerate}
		\item Two centers: Motown (Detroit, a.k.a. \textit{Hitsville}) and Stax (Memphis, a.k.a. \textit{Soulsville})
	\kl
\end{itemize}

\section{Motown}
\begin{itemize}
	\item \idate{1959}{motown-founded}, founder Berry Gordy
	\item The name comes from the nickname of Detroit - motor city - as it used to be a hub for industry
	\item Gordy worked in the automotive plants and envisioned a music label that operated in the same way as an assembly line
	\bl
		\item New in African-American music
		\item Had been done before but not all in one building
	\kl
	\item Songwriters
	\bl
		\item Holland/Dozier/Holland
		\item Smokey Robinson
	\kl
	\item Maxine Powell
	\bl
		\item Teacher at a 'finishing school' that taught the performers how to behave
		\item Wanted his artists to be able to perform in any place
	\kl
	\item Cholly Atkins
	\bl
		\item Performed the choreography, giving Motown a signature 'look'
	\kl
	\item The Funk Brothers: the house band
	\bl
		\item Loose collection of musicians who gave Motown its signature sound
	\kl
	\item Highly uniform product - the only difference between songs would be the singer
	\item Example: Smokey Robinson and the Miracles \song{Youd Better Shop Around}{1960}
	\bl
		\item Performance: restrained, polished presentation
		\item Lyrical content: shifted towards idealized romance
		\item Lip synced, no band behind them
	\kl
	\item Example: The Supremes \song{Stop in the Name of Love}{1965}
	\bl
		\item Illustrates the sound of Motown
		\item Beat and lyrics are very clear: made for dancing \textit{and} listening
		\item Diverse instrumentation (organ, vibraphone)
		\item Not the typical gospel vocals, smoother/breathy voice
		\item Clarity of sound (\textit{not} the wall of sound)
		\item Quality control
		\bl
			\item Motown regularly met to ensure their songs were similar quality to current top hits, hence \textbf{hitsville}
		\kl
	\kl
\end{itemize}

\section{Stax Records}
\begin{itemize}
	\item Founded by Jim \textbf{St}ewart and Estell \textbf{Ax}ton (siblings) in \idate{1959}{stax-founded}
	\item House band was Booker T and the M.G.s (\song{Green Onions}{1962}). They weren't invisible like Motown's band
	\item Collective decision making vs. Motown where Gordy always had the final decision
	\item Never used multi-track recording (outright refused it)
	\item More focused on energy than accuracy in recording
	\item Otis Redding joins in \idate{1965}{otis-redding-stax}
	\bl
		\item Example: \song{Try a Little Tenderness}{1966}
		\bl
			\item Cover of a TPA song
			\item Hit for Bing Crosby
			\item Very much Gospel influenced vocals
			\item Solid R\&B beat
			\item No elaborate arrangement
			\item Interacting with the band (not separated like motown)
			\item Band: Bar-Ks
			\item No elaborate choreography
			\item Not looking at the cameras (motown would always)
		\kl
		\item Tragically dies in plane crash \idate{1967}{otis-redding-dead}
	\kl
	\item Sam \& Dave
	\bl
		\item Example: \song{Soul Man}{1967}
		\bl
			\item Straightforward arrangement
			\item Strong Gospel vocals
			\item Interactions with the band, even in recording
			\item Mistake: brass forgot to play a note in verse 1 - Stax doesn't care since the energy is still good (\textbf{Soulsville})
		\kl
		\item Soul becomes a metaphor for black culture
		\item Still an optimistic sound - by 1967 this will take a turn
	\kl
\end{itemize}

\section{FAME}
\begin{itemize}
	\item Florence, Alabama Music Enterprise
	\item 1966 - Atlantic begins working with them by sending them artists to record
	\item House band: The Swampers
	\item They sign \textbf{Aretha Franklin}
	\bl
		\item Example: \song{Respect}{1967}
		\bl
			\item More militant sound
			\item Written by Otis Redding, originally about a relationship
			\item Becomes a protest song during the '65-'67 detroit riots
		\kl
	\kl
\end{itemize}

\section{Funk}
\begin{itemize}
	\item In the wake of the \idate{1968}{mlk-assassination} MLK assassination, there was push to re-africanize culture
	\item James Brown decides to invent Funk
	\bl
		\item He's been a star since the mid-late 50s
		\item First hit was \song{Please Please Please}{1956}, which was one of the first hits with soul characteristics
		\item Live album Live at the Apollo (1956) is considered to be the first album that reached 1M sales by an african american artist
		\item Papas got a Brand New Bag (1965) is a crossover hit for him
		\item Example: \song{I Feel Good}{1965}
		\bl
			\item Stax-like recording
		\kl
		\item Funk is based on 12-bar blues, with a typical composition of 2 12-bar blues separated by an AABA bridge made to appeal to white audiences.
		\item Cold Sweat (1967) introduces funk which Brown develops by reading a bit about African music traditions
		\item Example: \song{Get Up}{1970}
		\bl
			\item Not many chord changes - no longer important
			\item No clear melody
			\item Deprivileging harmony/melody in favor of rhythm
			\item Privilege to rhythm and its articulation
			\bl
				\item Tied to the re-africanization idea
				\item Interlocking groove based on African drum circles
				\item Brown mostly got it correct - there's a lead drummer but mostly an egalitarian group
			\kl
			\item Instrumentation is still pop, but played in a percussive manner
			\item Cyclical structure (vs. linear AABA or 12 bar blues)
			\item \textbf{'The One'} - the first down beat has to be very clear
			\item Another example of riff-based composition
		\kl
		\item Funk, beats, and loops lead to hip-hop
	\kl
\end{itemize}

\chapter{Folk and Psychedelic Rock}

\begin{itemize}
	\item By the '60s, teen fans of the golden age of rock 'n' roll are now in their early 20s, looking for more serious music. They find it in classical and folk music.
	\item Paradox of professional folk music
	\bl
		\item Folk positions itself \textit{against} the music industry
		\item Values tradition over innovation, in constrast to the industry
		\item So how can there be a professional folk musician who works in the industry?
	\kl
	\item Pre-WWII Folk Music
	\bl
		\item Explicitly policital, advocating for leftism, unions, etc.
		\item \textbf{Woody Guthrie}
		\bl
			\item 'This Machine Kills Fascists' - positions himself as another worker operting a machine
			\item Example \song{This Land is Your Land}{1940}
			\bl
				\item Recorded by the Weavers with \textbf{Pete Seeger}
				\item Weavers blacklisted in 1952
			\kl
		\kl
	\kl
	\item House Un-American Activities Committee (HUAC)
	\bl
		\item Ran from 1947-1956 under the oversight of McCarthy
		\item Targeted many folk musicians incluting Pete Seeger
		\bl
			\item Within a week of appearing in front of HUAC, Seeger's record deals, tours, etc all cancelled. He could not appear on TV until 1963
		\kl
	\kl
	\item As a result of the persecution (e.g. HUAC), folk falls off until the \textbf{folk revival}
	\bl
		\item Kingston Trio, Peter Paul \& Mary. All younger musicions, with a more produced/smoother sound
		\item Example \song{Tom Dooley}{1959} by the Kingston Trio
		\bl
			\item Demonstrates the smoother production of folk revival music
			\item Folk for people who grew up on the production techniques of golden age rock and roll
		\kl
	\kl
	\item Some musicians continue the old folk sound, for example \textbf{Bob Dylan}
	\bl
		\item Visits Guthrie (his idol) in New York
		\item '61-'65, Dylan is a traditional folk artist
		\bl
			\item Example \song{Hard Rain}{1962}
			\bl
				\item Numeric references give a biblical/ancient feel
				\item Written around the time of the Cuban missile crisis (1962)
			\kl
			\item Meets the Beatles in 1964, admires the reach they have which spurs on a change in his sound
		\kl
	\kl
	\item Dylan goes electric - \idate{1965}{dylan-electric}
	\bl
		\item At the Newport folk festival, plays electric instead of traditional folk thus inventing Folk Rock
		\item Very last minute decision
		\item Wasn't necessarily about acoustic vs. electric, it was about integrity to the genre and traditions
		\item Example \song{Like a Rolling Stone}{1965}
		\bl
			\item Basically signifies the invention of folk rock similarly to how James Brown did Funk
		\kl
	\kl
	\item Counter Culture: The Beats
	\bl
		\item Kerouac, Ginsberg
		\item Name comes from Jazz Beat, Beaten down, Beatitude (false consciousness)
		\item Beginning to move away from parent's ideology, questioning authority at a time when that was highly unusual
	\kl
	\item Rediscovery of Beat culture in the '60s
	\bl
		\item Greenwich Village NY, Haight-Asbury SF
		\item Sensory stimulation (psychedelics) to acheive higher consciousness
		\item Focus on stimulating neural activity = psychedelic
	\kl
	\item Counterculture music
	\bl
		\item Rock and Pop concerts become much louder - tied to the idea of psychedelia, overloading the senses to overcome false consciousness
		\item Lighting shows in concerts begin here: colors, motion, flashing, all synchronized to enhance neural activity
		\item Longer and unusual song forms
		\item Jamming/collective improvisation - the pattern is unknown even to the musicians
		\item Example: \song{Truckin'}{1970} by The Grateful Dead
		\bl
			\item Demonstrates the role of improvisation
			\item Adopts the process orientation of Dylan
			\item In live show, where the studio versions end the dead continue to improvise
		\kl
	\kl
	\item Acid/Psychedelic Rock
	\bl
		\item Jefferson Airplane. Example: \song{White Rabbit}{1967}
		\bl
			\item Unusually short song for this genre
			\item Ends up a hit, though that was not the goal
			\item Structure is a crescendo, mimicking an acid rush
		\kl
	\kl
	\item \idate{1967}{summer-of-love} - The peak of west-coast counterculture
	\item Beatitude became too inward focusing - needed an external goal and action
	\bl
		\item Counterculture becomes more politically active
		\item Vietnam war
		\item Yippies - Jerry Rubin and Abbie Hoffman
	\kl
\end{itemize}

\chapter{Hard Rock and Metal}

\section{70s and the Decline of Counterculture}
\begin{itemize}
	\item Psychedelic Blues - \textbf{Jimi Hendrix}
	\bl
		\item Guitarist in James Brown's band
		\item Becomes famous in England before returning to the U.S.
		\item Example \song{Voodoo Child}{1967}
		\bl
			\item Uses the wah-wah pedal
			\item Perfects existing techniques rather than invents new ones
			\item Whammy bar
			\item 12 bar verse structure
		\kl
	\kl
	\item Woodstock \idate{1969}{woodstock}
	\bl
		\item Free admission, aimed to sell film of the concert to make money
		\item 350-500,000 people show up when they expected about 50,000
		\item Bethel, NY
	\kl
	\item Altamont, California \idate{1969}{altamont}
	\bl
		\item The Rolling Stones missed out on Woodstock and try to do their own one
		\item Decide to give the Hell's Angels the security job and pay them with beer
		\item Armed with chains and lead-filled pool cues, the Hells Angels beat everyone up
		\item Massive failure, people die, ruins the image of the counterculture
	\kl
	\item Kent State Massacre \idate{1970}{kent-state} signals a sharp turn in the counter-culture
	\item in the 70's, a more cynical view develops given the failings of the counter culture, reinforced by
	\bl
		\item Eenergy crisis of 1973-1974 - first economic recession since WWII
		\item Watergate \idate{1974}{watergate}
		\item Fall of Saigon \idate{1975}{fall-saigon}
	\kl
	\item Soul music still popular, Stax/Motown no longer the hubs. A Philidelphia sound develops which eventually leads to disco. Example: \song{Love Train}{1972} by the OJays
	\item Funk grows in popularity
	\bl
		\item George Clinton
		\item Sly \& Family Stone, Parliament Funkadelic
		\item Example \song{Up for the Downstroke}{1974}
	\kl
	\item Pop becomes more mature as boomers grow into their '30s
	\bl
		\item Carol King (from Brill Building), James Taylor
		\item Example \song{It's Too Late}{1971}
		\bl
			\item "Adult contemporary" genre, new billboard chart for this genre
		\kl
	\kl
	\item 70's was the "Decade of the Album"
	\bl
		\item Eagles' Hotel California, Fleetwood Mac's Rumours
	\kl

\end{itemize}

\section{Emergence of Hard Rock and Metal}
\begin{itemize}
	\item Differences between Hard Rock and Metal
	\bl
		\item Metal - higher vocal distortion (e.g. growlers)
		\item Metal tends to be either very high or very low tempo, since there is more focus on musicianship. Also fewer tempo changes in Hard Rock
		\item Metal has a lot of western classical influences
		\item Lyrical themes - Hard Rock mostly influenced by golden age lyrics, while metal focuses on deeper themes like war, madness, and religion/mythology
	\kl
	\item Overall it's a spectrum, on one end is AC/DC and the other end is Metallica
	\item Founding bands of these genres all happen to be English - Led Zeppelin, Deep Purple, Black Sabbath
	\item Why all English? Counterculture of the 60s didn't really take hold in England so the economic cynicism reached there earlier
	\item Black Sabbath
	\bl
		\item Coming out of the British Blues revival
		\item Ozzy Osbourne wanted doomer music
		\item Example \song{War Pigs}{1970}
		\bl
			\item Lots of metal characteristics except for the very clear vocals
		\kl
	\kl
	\item Deep Purple
	\bl
		\item Jon Lord on electric organ through a guitar amp for distortion. The usage of an organ was unusual for a hard rock / heavy metal band
		\item Example \song{Highway Star}{1972}
		\bl
			\item Again characteristics of both
			\item Organ solo is very classical (incl. arpeggios) - Jon Lord was classically trained and one of the first classically trained artists to be in such a band
		\kl
	\kl
	\item Bands begin to have singular names and logos like corporations. No longer have the 'The' before the name and drop the plural. Reflects the turning inward of the counter culture


\end{itemize}

\chapter{Punk, Disco, and Hip-Hop}

\section{Punk}
\begin{itemize}
	\item One of a number of genres that rose in response to the success of hard rock and metal (and the hippie aesthetic)
	\item The Velvet Underground: New York, 1967
	\bl
		\item Lou Reed and John Cale, disliked the hippie aesthetic and how it helped the rise of the music industry
		\item Confrontational, Nihilistic
		\item Association with Andy Warhol - they were sort of his house band in his art studio
		\item Art of the everyday (Warhol)
	\kl
	\item Example \song{Heroin}{1967}
	\bl
		\item Stripped down, rejection of traditional approaches to instrumentation and songwriting
		\item Attempt to depict drug addiction in a realistic manner
		\item Simplistic structure, only 2 chords. Trying to get in touch with the audience rather than be separated like in hard rock and metal
	\kl
	\item CBGB club NYC
	\bl
		\item Bands like the Talking Heads, Blondie, Ramones
		\item Different sounds but all have a similar stripped down sound
		\item Almost like a return to the simplicity of Golden Age
		\item Ramones
		\bl
			\item DIY sound
			\item Three chords, and no solos (latter is too showy)
			\item Example \song{I Wanna Be Seedated}{1978}
			\bl
				\item Sounds almost like a '50s song
			\kl
			\item Became the model for English punk scene
		\kl
	\kl
	\item Sex Pistols
	\bl
		\item Malcolm McLaren
		\bl
			\item Promoter, wanted to bring American punk to England
			\item Urges the members to form a band
		\kl
		\item Johnny Rotten joins (rotten teeth)
		\item Epitomized DIY punk - reusing merch to create 'anti-merch'
		\item Example \song{God Save the Queen}{1977}
		\bl
			\item Comes out around the Queen's Silver Jubilee
			\item Example of how British punk was more politically minded than American
			\item Reached \#1 on the charts
			\bl
				\item Pressing workers in England wouldnt print it
				\item No one put the name on the charts, leaving an empty \#1 spot that month
			\kl
		\kl
		\item Break up at the end of their US tour of the south - got too famous
	\kl
	\item New Wave rises later on, it's like punk but more amenable to the establishment (Elvis Costello, The Cars). A somewhat more polished sound but more stripped down
\end{itemize}

\section{Disco}
\begin{itemize}
	\item In the early 70's, dance clubs form based on records instead of live bands
	\bl
		\item Particularly in the gay community in NYC, but it wasn't seen as marketable by the industry
		\item David Mancuso - "Invitation Only" (word of mouth) parties
		\bl
			\item Remixed soul/funk records using reel-to-reel recorders
			\item Created overall flow for the evening by mixing the songs
			\item Popularity of this type of party grows through the mid 70s
		\kl
		\item By 1972, Disco becomes a musical genre in its own right
		\item Soul record companies begin creating records oriented at being played in these kinds of clubs
		\bl
			\item No sudden changes in temp (all about 120bpm)
			\item Still sounded like soul mostly
		\kl
	\kl
	\item Example \song{Love Train}{1973} by the O'Jays
	\bl
		\item Has many Disco characteristics
		\item Clean production, no distortion
		\item Very steady, clear beat - complex arrangement on top
	\kl
	\item Example \song{Macho Man}{1978} by the Village people
	\item Disco also a response to stadium rock
	\bl
		\item Instead of a stage at front, there is a dance floor in the middle
		\item The audience is the one performing, not a band
	\kl
	\item Example \song{Le Freak}{1978} by Chic
	\bl
		\item No solos - there is a dance break though (the dancers do the solo)
	\kl
	\item Success of Disco peaks in 1977 - Saturday Night Fever. Dies out by 81
	\bl
		\item The film causes an explosion in disco clubs
		\item Rock backlash
		\bl
			\item "Disco Sucks" becomes a popular catchphrase
			\item Disco Demolition Night
			\bl
				\item July 12, 1979 Comisky Park Chicago
				\item Blew up a bunch of disco records
			\kl
			\item Why? Probably because it undermined the hippie aesthetic and put the musicianship on the backburner
		\kl
	\kl
	\item Disco lives on as House Music
	\bl
		\item Frankie Knuckles, an early disco DJ becomes disillusioned with its popularity and moves to Chicago
		\item In Chicago, Frankie starts performing at "The Warehouse" and plays a drum machine along with the records
		\item Remixes many funk and disco songs
		\item Becomes known as the "Warehouse Sound" or just "House Music"
	\kl
\end{itemize}


\section{Hip-Hop}
\begin{itemize}
	\item In the wake of the long hot summer of 1967, many wealthier people move out of the cities into the suburbs
	\item Bronx, 1970's especially hit by this migration
	\bl
		\item As a result, it falls into disrepair
		\item Expressway cuts through it, making it a forgotten place
		\item Youth begin to adopt a DIY approach to express themselves
		\bl
			\item Breakdancing - related to the dance break in disco
			\item Graffiti - walls, subway cars
			\item Music - turntable + records - rapping over instrumentals (usually parents' old soul/funk records)
		\kl
	\kl
	\item Precursors
	\bl
		\item Signifying and The Dozens
		\bl
			\item West african rhyming/word games
			\item Speed and improvisation is valued
			\item Signifying: storytelling
			\item The Dozens: insulting eachother
		\kl
		\item Example Signifyin' Monkey - virtuosic wordplay
		\item Jamaican Toasting
		\bl
			\item Sound system man - guy who has a system and records
			\item Until independence from Britain, radio in Jamaica was controlled by BBC and only played lame (british) music
			\item SSM only people with access to local music
			\item SSM became first generation of Jamaican record producers
			\bl
				\item Record to play at their yard parties
				\item Initially, same song on both sides of record
				\item Eventually, start using the second side to record some vocals of their own talking about how cool they are
			\kl
		\kl
	\kl
	\item Prehistory
	\bl
		\item Post independence, many Jamaicans move to NYC and the Bronx in particular, and bring their music
		\item Kool Herc (1973) - grew up listening to sound system men
		\bl
			\item Holds Jamaican-style yard parties in the Bronx
			\item Technique: Merry-Go-Round
			\bl
				\item Loop dance break on two copies of record so it would keep playing
				\item Does it in real time with two turntables
			\kl
		\kl
		\item Grand Master Flash (1976)
		\bl
			\item Develops Kool Herc's techniques, does them faster
			\item Beat matching - made Merry-Go-Round seamless by backspinning the record
		\kl
		\item Grand Master Melle Mel
		\bl
			\item Toaster/rapper for grandmaster flash who was busy working the records
			\item First to describe himself as an MC
			\item Does a full length rap
			\item First to write down the lyrics before performing
		\kl
	\kl
	\item Sugar Hill Records (1979)
	\bl
		\item Started by Sylvia Robinson (mildly famous former soul artist)
		\item Example \song{Rapper's Delight}{1980} by the Sugar Hill Gang
		\bl
			\item Based on "Good Times" by Chic
			\item First line: has to explain that he's not doing a mic test as most people wouldn't know what he's doing
		\kl
	\kl
	\item MTV goes on the air in 1981
	\bl
		\item Becomes the main source of major hits
		\item Songs without music videos struggle to become hits
		\item Problem for soul funk and hip-hop, MTV wouldn't play Black music
	\kl
	\item \song{Walk This Way}{1985} Aerosmith and Run DMC
	\bl
		\item Aerosmith is kind of washed by 1985, find out that the drums from walk this way are popular in hip-hop
		\item Release a music video that breaks through to MTV with Run DMC
	\kl
\end{itemize}

\end{document}