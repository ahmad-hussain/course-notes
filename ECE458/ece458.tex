\documentclass[11pt]{report}
\usepackage[margin=0.5in]{geometry}
\usepackage{mathpazo}
\usepackage{amsmath}
\usepackage[colorlinks=true]{hyperref}
\usepackage{parskip}
\usepackage{tikz}
\usepackage{circuitikz}
\tikzstyle{Si}=[circle,draw=blue,fill=blue!50,thick]
\usepackage{listings}

\lstset{
	basicstyle=\ttfamily,
	numbers=left}

% Make lists condensed %
\usepackage{enumitem}
\setlist{nosep}

\begin{document}

\chapter{Topic 1 - Security}
\section{CIA Model}
\begin{itemize}
	\item Confidentiality
	\begin{itemize}
		\item Cryptography: encryption and decryption
		\item Access Control: some policy that limits access to certain users through personal ID, etc.
		\item Entity Authentication: ensure the ID of someone
		\begin{itemize}
			\item Something they have (token, key)
			\item Something they know (password)
			\item Something they are (fingerprint)
		\end{itemize}
		\item Physical security
	\end{itemize}
	\item Integrity
	\begin{itemize}
		\item Backups, checksums, ECC
		\item Cryptography: MAC over the message or a digital signature using a private key (symmetric or asymmetric)
	\end{itemize}
	\item Availability
\end{itemize}

\section{Partial Plaintext Attack}
Suppose transactions in a RFID system are encrypted as $C = Enc(k, M)$ for some 4-digit PIN $k$. Then, if a plaintext $M_0$ and its encryption $C_0$ is known, then an attacker need only try all $10^4$ PINs to compute $C' = Enc(k_i, M_0)$ until $C' = C_0$, at which point they have the PIN. This is much less than the $10^4 \cdot 2^{|M|}$ that would normally be required.

\section{Buffer Overflow}
\begin{lstlisting}
void main() {
	char line[10];
	gets(line);
}
\end{lstlisting}
Writing more than 10 bytes to \texttt{line} will begin to overwrite (in order):
\begin{itemize}
	\item The old Frame Pointer of the caller
	\item The return address
	\item Local variables of the caller
\end{itemize}
In performing a buffer overflow attack, the goal is to overwrite that return address to point into some code that the attacker has inserted.

\section{Spectre Attack}
\begin{lstlisting}
if (x < array1_size)
	y = array2[array1[x] * 4096]
\end{lstlisting}

\begin{itemize}
	\item If \texttt{x >= array1\_size} and \texttt{array1[x]} contains (for example) a secret key, then if the conditions are right the CPU will perform the memory access in line 2 while checking the condition
	\begin{itemize}
		\item Conditions
		\begin{itemize}
			\item \texttt{array1\_size} and \texttt{array2\_size[k*4096]} are uncached
			\item The key \texttt{k=array1[x]} is cached
			\item The branch predictor assumes that the condition will likely be true (e.g. if many previous iterations were true, this will likely happen for most branch predictors)
		\end{itemize}
		\item The \texttt{array2} access will miss while the branch is still being checked, and the value \texttt{array2[k*4096]} will be placed in cache even though the condition was false
		\begin{itemize}
			\item The time it takes to access this memory can be measured dependent on \texttt{k*4096}, from which the key k can be derived
		\end{itemize}
	\end{itemize}
\end{itemize}

\section{Trusted Platform Module}
\begin{itemize}
	\item Standard for a secure co-processor
	\item Principle: root of trust and transitive trust
	\begin{itemize}
		\item A layer X has a set D(X) called its \textbf{dependencies}
		\item $L \in D(X)$ if at least one of the following is true:
		\begin{itemize}
			\item L has RW access to the data of X
			\item L has W access to the code of X
		\end{itemize}
		\item Notation: $L \xleftarrow{} X$ means $L \in D(X)$
		\begin{itemize}
			\item $\xleftarrow{}$ is transitive
		\end{itemize}
	\end{itemize}
	\item Secure boot
	\begin{itemize}
		\item Each segment $X_i$ verifies $X_{i+1}$ before passing execution to it
	\end{itemize}
	\item Authentication Authority
	\begin{itemize}
		\item Each layer has an AA that holds the secret key used to authenticate items in the layer above it
		\item So X in layer J can call the AA in layer $k < J$ to generate auth data for itself that it can send to a remote party for verification
		\begin{itemize}
			\item E.g. a digital signature using a secret key where the public key is known by the remote
		\end{itemize}
	\end{itemize}
\end{itemize}

\chapter{Topic 2 - Practical Cryptographic Schemes}
\section{Pseudo-Random Sequence Generators (PRSG)}\label{sec:PRSG}
\subsection{FSRs}
A feedback shift register is an n-bit register with bits $a_{n-1}...a_0$ where at each cycle the last bit is determined according to a feedback function $f(a_{n-1} ... a_0) = f(\vec{a})$. $(a_{n-1} ... a_0) \xleftarrow{} (a_n ... a_1)$ where $a_n = f(\vec{a})$. The output bit is $a_0$. If you picture it with the LSB on the right, the output is on the right and the MSB gets loaded with the feedback function on the left in each cycle.

\subsection{LFSR}\label{sec:LFSR}
An LFSR (Linear FSR) is an FSR where $f(\vec{a}) = \sum_{i = 0}^{n-1} c_ia_i$ and $c_i \in \{0, 1\}$.

An m-sequence (or maximal length or pseudo noise sequence) is the output sequence with the maximal period for an LFSR. For an N-bit LFSR this is $2^N-1$ bits long. An LFSR generates an m-sequence as its output if and only if its \textbf{characteristic polynomial} is \textbf{primitive}. LFSRs have the following properties:

\begin{enumerate}
	\item All output sequences are either periodic or \textit{eventually} periodic
	\item The minimal polynomial of an LFSR sequence is a divisor of its characteristic polynomial
	\item For a given LFSR, all its m-sequences are \textit{shift-equivalent} and \textit{shift-distinct} to all m-sequences \textbf{not} generated by the LFSR (i.e. by other LFSRs)
\end{enumerate}

To explain property 2, think of it in the following way. An LFSR A has some characteristic polynomial $f(x)$. It generates a set of sequences depending on the initial conditions. Consider one such sequence $\{a_i\}$. The minimal polynomial of this sequence is the \textit{lowest degree} polynomial that, if it were the characteristic polynomial $h(x)$ of an LFSR B, B would also generate $\{a_i\}$. Property 2 states that $f(x) | h(x)$ (i.e. $f(x) = p(x)h(x)$ for some $p(x)$). As a corollary, if $f(x)$ is \textit{irreducible}, it is the minimal polynomial of all the sequences it generates. Furthermore, the minimal period of any sequence $f(x)$ generates is equal to its period.


\subsection{Galois Fields}
A Galois Field is a \textbf{field} with a finite number of elements. All such fields with $p^n$ elements for prime $p$ (denoted GF($p^n$)) are isomorphic to $\mathbb{Z}_{p^n}$. The binary field GF(2) is unique with the elements $\{0, 1\}$: the operations * and + are a logical AND and XOR respectively.

\subsection{Polynomials over GF(2)}
The feedback function of an LFSR can be represented as a polynomial with coefficients in GF(2):
\begin{equation}
	\sum_{i=0}^{n-1}c_ix_i \leftrightarrow t(x) = \sum_{i=0}^{n-1}c_ix^i
\end{equation}

Such a polynomial is \textbf{irreducible} if it cannot be written as the product of two (non-constant) polynomials. For example $x^2-2$ is irreducible over the integers but not over the reals.

Note that we can define the \textbf{period} of a polynomial as the \textit{minimum} $r$ such that $f(x) | x^r-1$. A polynomial is \textbf{primitive} if it is irreducible and its period is $2^N-1$.
Note that primitive polynomials must have a constant term, otherwise we could factor out an x and it would thus be reducible.

\subsection{Correlation}
The cross correlation of two N-length binary sequences is given by:

\begin{equation}
	C_{a, b}(\tau) = \sum_{i=0}^{N-1} (-1)^{a_i + b_{i+\tau}}
\end{equation}

It measures how similar a shifted version of $b$ is to $a$, and is thus a function over $\tau$. When $a=b$ this is also called the \textbf{autocorrelation}. If the lengths are not the same, we can use the following notation (where $M$ is the length of b:

\begin{equation}
	\label{eqn:general-cross-correlation}
	C_{a^T,b}(\tau) = \sum_{i=0}^{T-1}(-1)^{a_i + b_{(i + \tau) \mod M}}
\end{equation}

\textbf{Note that the operation $+$ in \autoref{eqn:general-cross-correlation} is XOR}


\subsection{Linear Span Attack}
The degree of the minimal polynomial of a sequence \textbf{a} of length N is called its \textbf{linear span}. In other words, the linear span of the sequence is the minimal $l$ such that an $l$-bit LFSR with initial state $(a_0, a_1, ...a_{l-1})$ generates the sequence $\{a_0 ... a_N\}$.

Given \textbf{a} can we make an LFSR that generates it? Obviously $f(x) = x^N-1$ would work since we just load the whole sequence into the LFSR. Using the \textbf{Berlekamp-Massey algorithm}, we can construct the \textit{minimal} LFSR that generates \textbf{a}. \textit{By definition}, this LFSR will have degree equal to the linear span of \textbf{a}, denoted LS(\textbf{a}).

As it turns out, the Berlekamp-Massey algorithm only needs 2LS(\textbf{a}) bits to perform the construction. This means that if \textbf{a} has linear span $n$, and $2n < N$, we can use the BM algorithm to construct the LFSR and then run it to generate the remaining $N-2n$ bits. This is called a linear span attack.

\subsection{Nonlinear Generators}
The goal here is to maintain the randomness and efficiency that comes from using m-sequences of LFSRs as random bitstreams, while \textit{increasing} the linear span to avoid a linear span attack. In \textbf{filtering sequence generators} we feed the full LFSR state into a nonlinear function that serves as the output bit. In \textbf{combinatorial sequence generators} we feed the output bits of M LFSRs into a nonlinear function that serves as the output. In \textbf{clock controlled} generators we use some clocked FSM to control which output bits of LFSR1 get fed into LFSR2 and then to the output.

\subsection{Correlation Attack}
When we use a \textbf{combinatorial sequence generator}, we can perform this attack to recover the initial states of the $m$ LFSRs. These states are assumed to be the keys (e.g. a stream cipher scheme could be to place the key in the LFSRs and then use the output stream as a one-time pad on the data).

Let the size of LFSR $i$ be $n_i$. Let the output stream of LFSR $i$ be denoted $X_i$. Note that $0 \le i < m$. We assume that we have the output stream of $Z$, it has length $T$ and is denoted $z$. We can perform exhaustive search by trying all possible initial states and seeing if it generates Z. The complexity of this is
\begin{equation}
	T_0 = \prod_{i=0}^{m-1} 2^{n_i}
\end{equation}

Under some assumptions, we can improve this:
\begin{itemize}
	\item $X_i$ are iid binary random variables
	\item $Z$ is a random variable whose value is $h(X_0, X_1, ... X_{m-1})$
	\item $I_C \subseteq \{X_0, ... X_{m-1}\} \neq \emptyset$ is a set of LFSRs which, based on the structure of the overall PRSG and function $h$, we know to be \textbf{correlated} with $Z$
\end{itemize}

This last point is important; based on Kerchoff's principle, the structure of the PRSG is usually known (including the sizes and polynomials of all the LFSRs and the function $h$). It is only the key that is secret.

The main idea here is that if we know that an LFSR is correlated with the output sequence, we can find its initial state \textit{independent} of the states of the other LFSRs. We will \textbf{still be doing an exhaustive search} on the LFSR (by trying every possible shift of its m-sequence), but only on one LFSR at a time.

The other key idea is that if we know $P[Z = 0 | X_k = t] > \frac{1}{2}$ or $P[Z = 1 | X_k = t] > \frac{1}{2}$ for $t \in \{0, 1\}$, then for a sufficiently large sample size, we can assume that the initial state of LFSR$_k$ that generates the \textbf{highest} correlation between $Z$ and $X_k$ is most likely to be the correct one. In particular, the probability that this assumption is wrong goes to zero as the size of the output sequence we are checking goes to infinity.

With this, the steps to perform a correlation attack are as follows:

\begin{enumerate}
	\item Write the truth table for the combining function $h$ with inputs $X_i$ (the 0th bit of LFSR$_i$)
	\item Use the formula $P[Z=i|X_j=k] = \frac{P[Z=i \land X_j=k]}{P[X_j=k]} = \frac{\#[Z=i \land X_j=k]}{\#[X_j=k]}$ (where \# means the number of times the event shows up in the truth table) to compute all the conditional probabilities
	\item Decide the set $I_C$ using step 2 ($X_i$ such that $Z$'s value has a skewed distribution relative to the value of $X_i$)
	\item For each element $X_i \in I_C$
	\begin{enumerate}
		% TODO: how
		\item Figure out the m-sequence for this LFSR and call it $x_i$ (recall that this is unique for a given LFSR up to shift-equivalence. see \autoref{sec:LFSR})
		\item Compute $C_{z^T, x_i}(\tau)$ using \autoref{eqn:general-cross-correlation} for $\tau \in [0, 2^{n_i})$  (i.e. to check every possible shift)
		\item Compute $\tau_0 = \text{argmax}_\tau |C_{z^T, x_i}(\tau)|$. Set the initial state of LFSR$_i$ to $x_i$ shifted by $\tau_0$
	\end{enumerate}
\end{enumerate}

The complexity of a correlation attack is given by

\begin{equation}
	T_1 = \sum_{i=0}^{m-1}2^{n_i}
\end{equation}


\section{Stream Ciphers}
A stream cipher uses a key to generate a pseudorandom sequence based on the methods described in \autoref{sec:PRSG}. This sequence is then XORed with the data (which may be of arbitrary size, hence stream instead of block) to encrypt it, analogously to the one-time pad system.

\subsection{Operation}
Stream ciphers consist of two phases: \textbf{key initialization} and \textbf{PRSG running}. During key initialization, the initial vector IV and key K are mixed. The result is then passed as the initial value to the PRSG which begins outputting a key stream that is XORed with the plaintext. The decryption scheme does the \textbf{exact same thing} and therefore needs the same IV.

\subsection{RC4}
RC4 is meant for efficient software implementation. The key-initialization algorithm (\textbf{KIA}) uses the key K to generate a permutation of all $n$-bit integers in memory. The PRSG increments a value $i$ and uses $S[S[i] + S[i + S[i]]]$ modulo $2^n-1$ as the output block (then swaps the values it used). Yes it seems like nonsense because it was, but since the system was kept secret for a while no one knew. It was used in WEP which caused a huge security issue when the code was leaked and attacks were discovered.

\section{Block Ciphers}

A block cipher involves an encryption and decryption algorithm. Each algorithm is essentially a function which is a one-to-one mapping of $n$-bit vectors. $n$ is called the block size. The important property (correctness) is that $D \circ E = \mathbb{1}$ (that is, encrypting then decrypting a plaintext gives back the original message).

Two principles in designing a block cipher:
\begin{itemize}
	\item \textbf{confusion}: the statistical relationship between plaintext and ciphertext should depend on many or all bits of the key
	\item \textbf{diffusion}: each plaintext bit should affect multiple ciphertext bits (small change in input $\xrightarrow{}$ large change in output)
\end{itemize}

Examples of block ciphers include DES and AES. Common structures used in them include
\begin{itemize}
	\item S-boxes: substitution - essentially tabularly represented functions that map input blocks to output blocks
	\item Feistel structures - an NLFSR
\end{itemize}

Another useful way to think of a $n$-bit block cipher is as the set of all permutations over all $2^n$ $n$-bit vectors, where the key selects which permutation to use.

\subsection{Block Cipher Modes}
We need some way to use the block cipher on data that may not be equal to the block size. Modes are the way to achieve this.

\subsubsection{Electronic Codebook (ECB)}
Divide the input into block-sized units and run the block cipher on each one individually

\subsubsection{Cipher Block Chaining (CBC)}
ECB causes two equivalent blocks to have the same output, so CBC instead uses the output of one block as an initialization vector to the next.

\subsubsection{Cipher Feedback (CFB)}
Use the key to generate a keystream from ciphertext bits. $K_i$ = $Enc_{K}(C_{i-1})$ gets XORed with the plaintext to produce $C_i = M_i \oplus K_i$. This turns the block cipher into a stream cipher.

\subsubsection{Counter Mode (CTR)}
In CFB, instead of using the ciphertext as the input to the block cipher, use a counter which gets incremented for each bit.

\section{Birthday Attacks}

Given a set $S$ of size $n$, the probability that two of $m$ randomly chosen elements (with replacement) are the same is given by \autoref{eqn:birthday}

\begin{equation}
	\label{eqn:birthday}
	\approx 1 - e^{\frac{-m^2}{2n}} \le \frac{m^2}{2n}
\end{equation}

The practical result is that if an element can take $N$ different values, you can expect it to take on the same value after observing about $\sqrt{N}$ instances. That is, for an $n$-bit value, we can expect a collision after $\sqrt{2^n} = 2^\frac{n}{2}$ instances.

\subsection{Time-Memory Tradeoff Attack}
Also known as a \textit{meet-in-the-middle} attack. Suppose we know that some system is encrypting a fixed plaintext $p$ using a randomly generated key $k$ of $n$ bits (for example if every transaction in some banking system starts with encrypting 'hello'). Then we can choose $2^{\frac{n}{2}}$ random keys and store the encryption of $p$ under each of them. That is, we can compute the set $T = \{(k_i, Enc_{k_i}(p)) \ | \ i \in [0, 2^{\frac{n}{2}})\}$. We can expect to observe a key $k$ for which we have already precomputed $Enc_{k_i}(p)$ to be used after observing about $2^{\frac{n}{2}}$ transactions.

That is, we have traded off the memory of storing $T$ for the time of observing another $2^{\frac{n}{2}}$ transactions, and the time complexity becomes $2^{\frac{n}{2}}$.

\section{Security Models}
\subsection{Chosen Plaintext Attack (CPA)}
An attacker is given access to an encryption oracle that will provide $Enc(p)$ for any plaintext it is given. The goal of the attacker is to figure out which of two plaintexts $m_0$ and $m_1$ corresponds to a given ciphertext $c$. Note that since the attacker could ask for $Enc(p)$, no deterministic encryption can be CPA secure.

With CBC for example, any further queries of $Enc(m_0)$ and $Enc(m_1)$ will be different since the oracle holds some state.

\section{Hash Functions}
Three important properties of a hash function:

\begin{itemize}
	\item \textbf{Collision Resistance}: Finding $x$, $y$ s.t. $h(x) = h(y)$ is difficult
	\item \textbf{Second pre-image resistance}: Given $x$, finding $y$ s.t. $h(x) = h(y)$ is difficult
	\item \textbf{Pre-image resistance}: Given $z$, finding $x$ s.t. $h(x) = z$ is difficult
\end{itemize}

Note that satisfying the first property implies the other two are satisfied. The reverse is not true.

A hash function maps $n$ bits to $m$ bits where $n > m$, and thus multiple values may map to the same thing. The design of hash functions is thus similar to block ciphers but with the requirement of a one-to-one mapping being relaxed.


\section{Message Authentication Code (MAC)}
A MAC is used in symmetric-key cryptography to verify that a message was sent by someone who knows the key $k$. There are many ways to implement it using keyed hash functions, block ciphers, or authenticated encryption. The key property is \textbf{unforgeability} - even if an attacker has access to an oracle that produces $t = MAC_{k}(m)$, it cannot produce $(m, t)$ such that Verif$(m,t) = 1$ and $m$ was not part of a query to the oracle.

\section{Public Key Cryptography}
\subsection{One-Way Functions}
A fundamental concept in public key cryptography is that of a one-way function. A function is 'one-way' if it is 'easy' to compute its output, but difficult to compute the input given its output. A \textit{trapdoor} one-way function is a one-way function that, if one knows some special piece of information, makes it easy to compute the input given the output.


\subsection{Diffie-Hellman Key Exchange}
The setting for Diffie-Hellman key exchange is a finite field $\mathbb{F}_p$ with $p$ elements. $g$ is a 'primitive element' of the field; that is, each nonzero element of $\mathbb{F}_p$ can be written as $g^i$ for some integer $i$. Note that in the finite field, the addition and multiplication operations are done modulo $p$ but this will often be omitted for brevity. This is equivalent to:

\begin{align}
	\label{eqn:primitivity-1}
	g^{p-1} &= 1\\
	\label{eqn:primitivity-2}
	g^{i} &\neq 1 \ \ \ \ \ \ \forall \ \ 1 \le i < p - 1
\end{align}

Each user $i$ has a keypair $(i, g^i)$. By the discrete log problem, $i$ is difficult to compute given $g^i$ (but not vice versa). Two users $a$ and $b$ can compute a shared key $k$ by sending eachother their public key. Then, user $a$ can compute $g^{ab} = (g^b)^a$ using their private key and user $b$ can do the same. Then the shared key is $g^{ab}$ which can be used to perform symmetric encryption/authentication.

\subsubsection{Lagrange's Theorem}
\autoref{eqn:primitivity-2} requires us to check $p-1$ values. Instead, we can check if an element is primitive using Lagrange's theorem. Simply put, an element $a$ is primitive if $a^{p-1}=1$ and $a^{q} \neq 1$ for all prime factors $q$ of $p-1$.

For example, if $p=47$ then since $(p-1) = 46 = 23\cdot2$ we only need to check the following to verify if $a$ is primitive:

\begin{align*}
	a^{p-1} &= a^{46} = 1\\
	a^{23} &\neq 1\\
	a^2 &\neq 1\\
\end{align*}

\subsection{RSA}
RSA relies on the difficulty of factoring large numbers. $\phi(n)$ is the Euler totient function and is equal to the number of integers up to $n$ that are relatively prime to it (two numbers $a, b$ are relatively prime if $\text{gcd}(a,b) = 1$). Obviously, $\phi(p) = p-1$ for prime numbers $p$ (i.e. all except itself). It also turns out that for the product of two prime numbers $p, q$: $\phi(p\cdot q) = (p-1)(q-1)$.

\subsubsection{Key Generation}

The process to generate the public and private keypair is as follows:

\begin{enumerate}
	\item Generate large primes $p$ and $q$
	\item Compute $n = pq$ and $\phi(n) = (p-1)(q-1)$
	\item Select the \textbf{public exponent} $e \in [1, \phi(n)-1]$ such that $e$ is relatively prime to $\phi(n)$
	\item Compute $d = e^{-1} \mod{\phi(n)}$
\end{enumerate}

The \textbf{public key} is the tuple $(n, e)$. The \textbf{private key} is the tuple $(d, p, q)$.

Anyone who knows $n, e$ cannot easily find $\phi(n)$ and thus can't compute $d$.

\subsubsection{Encryption}
A plaintext $m < n$ can be encrypted as $c = m^e \mod{n}$ using info from the public key.

\subsubsection{Decryption}
A ciphertext $c$ can be decrypted as $m = c^d \mod{n}$.

\subsubsection{Signature (RSA-DSA)}
A message can be signed by computing $h(m)^d \mod{n}$ for some hash function $h$ that outputs a suitably sized value. This can be verified by computing $s^e \mod{n}$ and comparing it with $h(m)$.

\subsubsection{Computing d}
 We use the extended euclidean algorithm to find $d = e^{-1}\mod{\phi(n)}$ given $e$ and $\phi(n)$. It gives us $d, k$ such that $de + k\phi(n) = 1$ and this gives us $d$.

\section{Digital Signature Standard (DSS)}
This scheme only provides signatures. It has the following elements:

\begin{enumerate}
	\item $p$ a (large) prime number
	\item $q$ a prime factor of $(p-1)$
	\item $g \in \mathbb{F}_p$ with $\text{ord}(g) = q$
	\item $h$ a cryptographic hash function
\end{enumerate}


\subsection{Keypair}
Signer selects some $x \in [1, q-1]$ as the private key. $y = g^x$ is the public key.

\subsection{Signature}
The signer performs the following for each message to sign:

\begin{enumerate}
	\item Randomly select $k \in [1, q-1]$
	\item Compute $r = g^k$
	\item Solve for $s$ in the signing equation (\autoref{eqn:dss-signing-eqn})
\end{enumerate}

\begin{equation}
	\label{eqn:dss-signing-eqn}
	h(m) \equiv xr + ks \mod{q}
\end{equation}

The signature is provided as the tuple $(r, s)$. Note how steps 1 and 2 reflect the keygen process. Step 3 requires the usage of the extended euclidean algorithm.

\subsection{Verification}
The verifier does the following:

\begin{enumerate}
	\item Compute $u = h(m)s^{-1} \mod{q}$ and $v = -rs^{-1}\mod{q}$
	\item Compute $w = (g^uy^v \mod{p})\mod{q}$
	\item Accept if $w = r$
\end{enumerate}

This works because of the following. Starting with \autoref{eqn:dss-signing-eqn} (and recalling that $r = g^k$):

\begin{align*}
	k &= h(m)s^{-1} - xrs^{-1}\\
	g^k &= g^{h(m)s^{-1} - xrs^{-1}}\\
	r &= g^{u}(g^x)^{-rs^{-1}}\\
	r &= g^u y^v\\
\end{align*}

% TODO
\section{Elliptic Curve Cryptography}

An elliptic curve is a set of points $(x, y) \in F$ that satisfy some cubic equation. We can define an addition operation on this set of points to form a \textbf{group}: to add two points $P$ and $Q$ draw a line between them and set $R = (x, -y)$ where $(x, y)$ is the third point that intersects the curve (note that the $y$ coordinate is flipped). If $P=Q$ then the line connecting them is just the tangent at $P$.

\subsection{ECC Finite Fields}
We restrict our attention to points where $x, y \in \mathbb{F}_p$. Given parameters $(a, b, p)$ we can solve for all such points on the curve as follows:

\begin{enumerate}
	\item Compute all the 'quadratic residues' in $\mathbb{F}_p$. That is, the set $QR = \{i | \exists y \in \mathbb{F}_p \text{ s.t. } i = y^2 \mod{p} \}$
	\item For all $x \in \mathbb{F}_p$, compute $x^3 + ax + b$ and see if $x \in QR$. If so, then $(x, y)$ is a point where $y^2 = x^3 + ax + b$.
\end{enumerate}

The group formed by these points is \textbf{cyclic}, that is every element is a generator. Do note that there is technically also a 'point at infinity' denoted $\infty$ or $\mathcal{O}$ which is not a generator which acts as the additive identity.

\subsection{Keypair}
A curve and field is selected, along with an arbitrary generator point $P$. The \textbf{private key} is a random integer $d$ where $0 < d < q$. The \textbf{public key} is $Q = dP$.

\subsection{EC-DH}
We can compute a shared diffie-hellman secret using $d_AQ_B = d_BQ_A = (d_Ad_B)P$.


\end{document}
